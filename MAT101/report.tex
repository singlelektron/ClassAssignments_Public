\documentclass{beamer}
\usepackage{tikz}
\usetheme{Madrid}  % 主题
\usecolortheme{default}  % 颜色主题

% \usepackage{amsmath,amssymb,enumerate,amsbsy,amsthm,graphicx,titlesec,color}

\title{Solving Polynomial Equations Using Complex Roots}
\author{ZHANG YANG MAT2509036}
\date{\today}

\begin{document}

\begin{frame}
    \titlepage
    \begin{tikzpicture}[remember picture, overlay]
        \node[anchor=south west, inner sep=8pt] at (current page.south west) {
            \includegraphics[width=3cm]{xmumlogo.png}
        };
    \end{tikzpicture}
\end{frame}

\begin{frame}{Catalogue}
    \tableofcontents
\end{frame}

\section{Fundamental Theorem of Algebra}
\begin{frame}{Fundamental Theorem of Algebra}
    \begin{block}{Theorem 1}
        Let \(P(z) = a_{n}z^{n}+a_{n-1}z^{n-1}+\cdots +a_{1}z+a_{0}\) be a polynomial of degree n with coefficient in \(\mathbb{C}\).
        Then the polynomial equation \(P(z) = 0\) has a root in \(\mathbb{C}\).
    \end{block}
\end{frame}

\begin{frame}{Corollary}
    \begin{block}{Theorem 2}
        Let \(P(z) = a_{n}z^{n}+a_{n-1}z^{n-1}+\cdots +a_{1}z+a_{0}\) be a polynomial of degree n with coefficient in \(\mathbb{C}\).
        Then the polynomial equation \(P(z) = 0\) has n roots in \(\mathbb{C}\), counting multiplicities.
    \end{block}
\end{frame}

\begin{frame}{Theorem}
    \begin{block}{Theorem 3}
        Let \(P(z)\) be a polynomial of degree n with coefficients in \(\mathbb{R}\).
        If \(\alpha = a + bi\) is a root of \(P(z) = 0\) with \(b \neq  0\), then \(\overline{\alpha}  = a - bi\) is also a root of \(P(z) = 0\).
    \end{block}
\end{frame}


\section{Viète Theorem}

\begin{frame}{Viète Theorem}
    \begin{block}{Theorem}
        Let \(z_{1},z{2},\cdots,z_{n}\) be the n roots of the equation \(a_{n}z^{n}+a_{n-1}z^{n-1}+\cdots +a_{1}z+a_{0} = 0\), where \(a_{n}\neq0\). Then\\
        \begin{align}
            z_{1}+z_{2}+\cdots+z_{n} &= -\frac{a_{n-1}}{a_{n}} \nonumber \\
            \sum_{1\leqslant j<k\leqslant n}  &= \frac{a_{n-2}}{a_{n}} \nonumber \\
            &\vdots \nonumber \\
            \prod_{j=1}^{n}z_{j} &= (-1)^{n}\frac{a_{0}}{a_{n}} \nonumber
        \end{align}
    \end{block}
\end{frame}

\section{Examples of solving Polynomial Equations}
\begin{frame}{Examples}
    \begin{block}{Example 1}
        Given that \(2-i\) is a root of the equation \(z^{4}-3z^{2}+az+40 = 0\), where \(a \in \mathbb{R} \). Find the value of \(a\) and the other roots of the equation.\\
    \end{block}
\end{frame}

\begin{frame}{Examples}
    \begin{block}{Example 1}
        Given that \(2-i\) is a root of the equation \(z^{4}-3z^{2}+az+40 = 0\), where \(a \in \mathbb{R} \). Find the value of \(a\) and the other roots of the equation.\\
    \end{block}

    \begin{block}{Answer}
        Since \(2-i\) is a root, \(2+i\) is a root.
        \begin{align}
            P(z) &= [z-(2-i)][z-(2+i)]Q(z) \nonumber \\
                 &= (z^{2}-4z+5)Q(z) \nonumber \\
                 &= (z^{2}-4z+5)(z^{2}+4z+8)+(a+12)z \nonumber \\
                 &= [z-(2-i)][z-(2+i)][z-(-2+i)][z-(-2-i)] \nonumber
        \end{align}
        So \(a = -12\) and \(z = 2 \pm i, -2\pm i\)
    \end{block}
\end{frame}

\begin{frame}{Examples}
    \begin{block}{Example 2}
        Given that the three roots of the equation \(z^{3}+6z^{2}+4z-3 = 0\) are \(\alpha, \beta\), and \(\gamma \). Find \(\alpha +\beta +\gamma \).\\
    \end{block}

    \begin{block}{Answer}
        Apply the Viete Theorem to get
        \begin{align}
            \alpha+ \beta +\gamma &= - \frac{a_{n-1}}{a_{n}} \nonumber \\
            &= - \frac{6}{-1} = -6 \nonumber
        \end{align}
    \end{block}
\end{frame}

\begin{frame}
    \titlepage
    \begin{tikzpicture}[remember picture, overlay]
        \node[anchor=south west, inner sep=8pt] at (current page.south west) {
            \includegraphics[width=3cm]{xmumlogo.png}
        };
    \end{tikzpicture}
\end{frame}

\end{document}